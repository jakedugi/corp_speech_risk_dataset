
\documentclass[11pt]{article}
\usepackage[utf8]{inputenc}
\usepackage{geometry}
\usepackage{graphicx}
\usepackage{booktabs}
\usepackage{array}
\usepackage{multirow}
\usepackage{float}
\usepackage{amsmath}
\usepackage{amssymb}

\geometry{margin=1in}

\title{Corporate Speech Risk Dataset: Comprehensive Analysis}
\author{Dataset Characteristics and Statistics}
\date{\today}

\begin{document}

\maketitle

\section{Dataset Overview}

This document presents a comprehensive analysis of the final clean dataset used for corporate speech risk modeling. The dataset consists of legal case records with associated monetary outcomes, filtered to ensure data quality and relevance for litigation risk prediction.

\subsection{Core Statistics}

\begin{table}[H]
\centering
\caption{Core Dataset Statistics}
\begin{tabular}{lr}
\toprule
\textbf{Metric} & \textbf{Value} \\
\midrule
Total Quotes & 24,462 \\
Total Cases & 125 \\
Outcome Range & \$31,764 -- \$5,000,000,000 \\
Median Outcome & \$4,000,000 \\
\bottomrule
\end{tabular}
\end{table}

\subsection{Token Analysis}

\begin{table}[H]
\centering
\caption{Token Count Statistics}
\begin{tabular}{lr}
\toprule
\textbf{Statistic} & \textbf{Tokens per Record} \\
\midrule
Mean & 15.7 \\
Median & 12.0 \\
Minimum & 5 \\
Maximum & 121 \\
25th Percentile & 8.0 \\
75th Percentile & 20.0 \\
Standard Deviation & 12.0 \\
\bottomrule
\end{tabular}
\end{table}

\subsection{Case Size Analysis}

\begin{table}[H]
\centering
\caption{Case Size Statistics (Records per Case)}
\begin{tabular}{lr}
\toprule
\textbf{Statistic} & \textbf{Records per Case} \\
\midrule
Mean & 195.7 \\
Median & 33.0 \\
Minimum & 1 \\
Maximum & 4,620 \\
25th Percentile & 5.0 \\
75th Percentile & 110.0 \\
Standard Deviation & 550.4 \\
\bottomrule
\end{tabular}
\end{table}

\section{Label Distribution and Support}

\subsection{Outcome Bin Distribution}

The dataset uses stratified outcome bins based on monetary judgments, creating three equally-sized tertile bins for ordinal risk modeling. \textbf{Important}: Case distribution is balanced by design (~33.3\% each), quote distribution is significantly imbalanced due to high-outcome cases containing substantially more quotes per case.

\begin{table}[H]
\centering
\caption{Support Statistics by Outcome Bin}
\begin{tabular}{lrrrrr}
\toprule
\textbf{Bin} & \textbf{Cases} & \textbf{\% Cases} & \textbf{Quotes} & \textbf{\% Quotes} & \textbf{Mean Quotes/Case} \\
\midrule
Low & 43 & 34.4\% & 3595 & 14.7\% & 83.6 \\
Medium & 38 & 30.4\% & 4026 & 16.5\% & 105.9 \\
High & 44 & 35.2\% & 16841 & 68.8\% & 382.8 \\

\bottomrule
\end{tabular}
\end{table}

\section{Outcome Distribution Analysis}

\subsection{Comprehensive Outcome Statistics}

The following table provides detailed statistical analysis of the real monetary outcomes, revealing the distribution characteristics that inform our modeling approach.

\begin{table}[H]
\centering
\caption{Comprehensive Real Outcome Distribution Statistics}
\begin{tabular}{lr}
\toprule
\textbf{Statistic} & \textbf{Value (USD)} \\
\midrule
Minimum & \$31,764 \\
5th Percentile & \$50,000 \\
10th Percentile & \$75,000 \\
25th Percentile (Q1) & \$250,000 \\
Median (Q2) & \$4,000,000 \\
Mean & \$200,973,309 \\
75th Percentile (Q3) & \$35,000,000 \\
90th Percentile & \$184,886,652 \\
95th Percentile & \$1,000,000,000 \\
99th Percentile & \$5,000,000,000 \\
Maximum & \$5,000,000,000 \\
\midrule
Standard Deviation & \$799,187,411 \\
Skewness & 5.32 \\
Kurtosis & 28.47 \\
\bottomrule
\end{tabular}
\end{table}

\subsection{Tertile Boundary Analysis (33/33/33 Split)}

The following table shows the exact dollar thresholds that define our three equally-sized outcome bins, demonstrating the dynamic tertile-based stratification approach.

\begin{table}[H]
\centering
\caption{Tertile Boundary Definitions and Bin Characteristics}
\begin{tabular}{lrr}
\toprule
\textbf{Bin} & \textbf{Range (USD)} & \textbf{Cases} \\
\midrule
Low (bin\_0) & \$31,764 -- \$710,258 & 43 \\
Medium (bin\_1) & \$710,258 -- \$9,600,000 & 38 \\
High (bin\_2) & \$9,600,000 -- \$5,000,000,000 & 44 \\
\bottomrule
\end{tabular}
\end{table}

\subsection{Within-Bin Distribution Analysis}

Detailed analysis of outcome distributions within each tertile bin reveals the internal structure and economic significance of each risk category.

\begin{table}[H]
\centering
\caption{Detailed Statistics by Outcome Bin}
\begin{tabular}{lrrr}
\toprule
\textbf{Statistic} & \textbf{Low Bin} & \textbf{Medium Bin} & \textbf{High Bin} \\
\midrule
Mean Outcome & \$179,386 & \$3,891,549 & \$567,410,708 \\
Median Outcome & \$75,000 & \$3,773,578 & \$85,000,000 \\
Std Deviation & \$169,825 & \$2,399,384 & \$1,267,778,439 \\
Range Span & \$583,236 & \$8,789,742 & \$4,990,400,000 \\
Total Value & \$7,713,604 & \$147,878,854 & \$24,966,071,157 \\
\% of Total Value & 0.0\% & 0.6\% & 99.4\% \\
\bottomrule
\end{tabular}
\end{table}

\subsection{Real Outcome Distribution}

The continuous outcome distribution reveals the expected skew in corporate litigation, justifying our quantile-based binning approach for statistical robustness.

\begin{figure}[H]
\centering
\includegraphics[width=0.8\textwidth]{figures/outcome_distribution_log.pdf}
\caption{Log-scale distribution of case outcomes showing natural skew in corporate litigation damages. Vertical lines indicate quantile bin boundaries and central tendencies. The log transformation reveals the underlying distribution structure and justifies binning for ordinal modeling.}
\end{figure}

\subsection{Label Distribution Figures}

\begin{figure}[H]
\centering
\includegraphics[width=0.7\textwidth]{figures/case_label_distribution.pdf}
\caption{Distribution of cases across outcome bins. Shows balanced stratification with slight skew toward low-outcome cases.}
\end{figure}

\begin{figure}[H]
\centering
\includegraphics[width=0.7\textwidth]{figures/quote_label_distribution.pdf}
\caption{Distribution of quotes across outcome bins. Quote distribution may differ from case distribution due to varying case sizes.}
\end{figure}

\section{Case and Record Length Analysis}

\begin{figure}[H]
\centering
\includegraphics[width=0.8\textwidth]{figures/case_sizes_histogram.pdf}
\caption{Distribution of case sizes measured in records per case. Shows the variability in case complexity and litigation scope.}
\end{figure}

\begin{figure}[H]
\centering
\includegraphics[width=0.8\textwidth]{figures/token_counts_histogram.pdf}
\caption{Distribution of record lengths measured in tokens per record. Indicates the typical length of individual quotes or statements.}
\end{figure}

\section{Temporal Coverage and Distribution}

\subsection{Case Year Distribution}

The dataset spans multiple years of corporate litigation, providing temporal diversity important for robust model evaluation and understanding litigation trends over time.

\\begin{table}[H]
\\centering
\\caption{Distribution of Cases by Year}
\\begin{tabular}{rrr}
\\toprule
\\textbf{Year} & \\textbf{Cases} & \\textbf{Percentage} \\\\
\\midrule
2000 & 1 & 0.8\% \\
2005 & 2 & 1.6\% \\
2006 & 2 & 1.6\% \\
2007 & 2 & 1.6\% \\
2008 & 4 & 3.2\% \\
2009 & 3 & 2.4\% \\
2010 & 2 & 1.6\% \\
2011 & 3 & 2.4\% \\
2012 & 4 & 3.2\% \\
2013 & 2 & 1.6\% \\
2014 & 7 & 5.6\% \\
2015 & 3 & 2.4\% \\
2016 & 8 & 6.4\% \\
2017 & 7 & 5.6\% \\
2018 & 6 & 4.8\% \\
2019 & 6 & 4.8\% \\
2020 & 8 & 6.4\% \\
2021 & 13 & 10.4\% \\
2022 & 11 & 8.8\% \\
2023 & 16 & 12.8\% \\
2024 & 9 & 7.2\% \\
2025 & 6 & 4.8\% \\

\\bottomrule
\\end{tabular}
\\end{table}

\\begin{figure}[H]
\\centering
\\includegraphics[width=0.9\\textwidth]{figures/temporal_distribution.pdf}
\\caption{Temporal distribution of cases by year showing the dataset's coverage across multiple litigation periods. The distribution reveals patterns in corporate litigation activity and ensures temporal diversity for robust model evaluation.}
\\end{figure}

\\subsection{Temporal Coverage Summary}

\\begin{itemize}
\\item \\textbf{Year Range}: 2000-2025 (26 years)
\\item \\textbf{Cases with Extractable Years}: 125 / 125 (100.0\%)
\\item \\textbf{Average Cases per Year}: 5.7
\\item \\textbf{Peak Litigation Year}: 2023 (16 cases)
\\end{itemize}

\\section{Jurisdictional Coverage and Context}

\\subsection{Court Analysis with Outcome Values}

The following table shows both case representation and economic impact by jurisdiction, revealing potential jurisdictional biases in litigation outcomes.

\begin{table}[H]
\centering
\caption{Top 5 Courts: Case Count and Outcome Value Analysis}
\begin{tabular}{lrrr}
\toprule
\textbf{Court} & \textbf{\% Cases} & \textbf{\% Quotes} & \textbf{\% Total Outcome Value} \\
\midrule
CAND & 17.6\% & 30.7\% & 25.1\% \\
NYSD & 8.8\% & 13.7\% & 10.1\% \\
DED & 2.4\% & 13.1\% & 1.0\% \\
TXSD & 1.6\% & 7.6\% & 0.1\% \\
MND & 1.6\% & 6.7\% & 0.1\% \\

\bottomrule
\end{tabular}
\end{table}

\begin{figure}[H]
\centering
\includegraphics[width=0.9\textwidth]{figures/top_courts.pdf}
\caption{Top 5 courts by quote count. Shows jurisdictional coverage with concentration in major federal districts.}
\end{figure}

\subsection{State Distribution and Risk Profile}

Geographic analysis reveals state-level patterns in litigation outcomes, important for corporate risk assessment and insurance modeling.

\begin{figure}[H]
\centering
\includegraphics[width=0.9\textwidth]{figures/top_states.pdf}
\caption{Top 5 states by quote count. Reflects geographic distribution of corporate litigation activity.}
\end{figure}

\begin{figure}[H]
\centering
\includegraphics[width=0.9\textwidth]{figures/state_bin_crosstab.pdf}
\caption{State vs. outcome bin cross-tabulation. Shows jurisdictional risk patterns with some states showing higher concentrations of high-outcome cases, relevant for corporate monitoring and insurance applications.}
\end{figure}

\section{Speaker Analysis and Diversity}

\subsection{Speaker Distribution}

\begin{figure}[H]
\centering
\includegraphics[width=0.9\textwidth]{figures/top_speakers.pdf}
\caption{Top 5 speakers by quote count after filtering. Shows the most frequently quoted entities in corporate litigation contexts.}
\end{figure}

\subsection{Speaker Concentration Metrics}

Analysis of speaker diversity reveals dataset representativeness and potential concentration bias.

\begin{table}[H]
\centering
\caption{Speaker Diversity and Concentration Analysis}
\begin{tabular}{lr}
\toprule
\textbf{Metric} & \textbf{Value} \\
\midrule
Total Unique Speakers & 8876 \\
Gini Coefficient & -0.457 \\
Herfindahl-Hirschman Index (HHI) & 24 \\
Top 5 Speaker Concentration & 8.0\% \\
\bottomrule
\end{tabular}
\end{table}

\textbf{Interpretation:}
\begin{itemize}
\item \textbf{Gini Coefficient} (0 = perfect equality, 1 = maximum inequality): -0.457 indicates moderate concentration
\item \textbf{HHI} (0-10000 scale): 24 suggests competitive speaker distribution
\item \textbf{Top 5 Concentration}: 8.0\% of quotes from top 5 speakers
\end{itemize}

\section{Data Quality and Filtering Justifications}

\subsection{Comprehensive Filtering Impact Analysis}

The final dataset reflects several principled filtering decisions to ensure data quality and modeling relevance. The following table quantifies the impact of each filtering criterion:

\begin{table}[H]
\centering
\caption{Filtering Criteria and Impact Assessment}
\begin{tabular}{p{3cm}p{6cm}rr}
\toprule
\textbf{Criterion} & \textbf{Rationale} & \textbf{Cases Removed} & \textbf{\% Impact} \\
\midrule
Missing Outcomes & Ensure supervised learning feasibility; focus on cases with quantifiable litigation risk & 140 & 51.3\% \\
Outlier Threshold (\$10B+) & Remove extreme outliers that distort ordinal binning and model calibration & 2 & 0.7\% \\
Speaker Filtering & Focus on defendant corporate speech; exclude judicial/regulatory commentary for interpretability & 0 & 0.0\% \\
\midrule
\textbf{Total Retained} & \textbf{Final clean dataset} & 125 & 45.8\% \\
\bottomrule
\end{tabular}
\end{table}

\subsection{Filtering Justifications}

\begin{itemize}
\item \textbf{Missing Outcomes} (140 cases removed): Ensures supervised learning feasibility and focuses analysis on cases with quantifiable litigation outcomes
\item \textbf{Outlier Threshold \$10B+} (2 cases removed): Prevents extreme outliers from distorting ordinal bin boundaries and model calibration while retaining representativeness
\item \textbf{Speaker Filtering}: Excludes non-defendant speakers (Court, FTC, Plaintiff, State, Commission, Congress, Circuit, FDA) to focus on corporate speech risk and improve label interpretability
\item \textbf{Case-Level Integrity}: Maintained complete case groupings throughout to prevent data leakage in cross-validation evaluation
\end{itemize}

\subsection{Stratification Strategy}

The three-bin stratification approach balances several modeling considerations:
\begin{itemize}
\item \textbf{Ordinal Structure}: Preserves natural ordering of litigation severity
\item \textbf{Statistical Power}: Ensures sufficient support in each bin for robust evaluation
\item \textbf{Interpretability}: Maps to intuitive risk categories (Low/Medium/High)
\item \textbf{Fairness}: Quantile-based binning prevents outcome magnitude bias
\end{itemize}



\section{Rolling-Origin Temporal Cross-Validation Analysis}

\subsection{Methodology Overview}

The dataset employs rolling-origin temporal cross-validation to ensure robust model evaluation while maintaining case-level integrity and temporal purity. This approach groups all quotes from the same legal case together, preventing data leakage, while using temporal ordering to prevent future information from influencing past predictions.

\subsubsection{Key Methodological Features}
\begin{itemize}
\item \textbf{Case-Level Grouping}: All quotes from a single case remain in the same fold (zero case bleed)
\item \textbf{Temporal Purity}: Rolling-origin design where training data temporally precedes evaluation data in each fold
\item \textbf{Train-Only Tertiles}: Outcome bins (Low/Medium/High) calculated using training data only per fold to ensure true temporal boundaries
\item \textbf{Boundary Convention}: Low: $y < e_1$; Medium: $e_1 \leq y \leq e_2$; High: $y > e_2$, where edges $e_1, e_2$ are computed on TRAIN-only per fold
\item \textbf{Support Weighting}: Case size handled via inverse-$\sqrt{N}$ support weights (clipped to [0.25, 4.0], normalized per fold) rather than stratification
\item \textbf{Split Naming}: Within folds: Train/Validation (VAL); Final tuning: DEV; Final holdout: OOF Test
\item \textbf{Class Weighting}: Compensates for quote-level imbalance through inverse frequency weighting combined with case support weights
\item \textbf{Range Coverage}: Each fold covers the full spectrum of monetary outcomes within its temporal window
\end{itemize}

\subsection{Class Weight Analysis}

To address the significant quote-level imbalance (14.9\% low, 16.5\% medium, 68.6\% high), we employ class weighting based on inverse frequency normalization combined with case-level support weighting.

\textbf{Important Note}: The quote distribution appears highly imbalanced by design. Our methodology balances at the case level during binning (achieving 33/33/33 case distribution per training fold), then handles quote-level imbalance through the combined weighting scheme described below.

\begin{table}[H]
\centering
\caption{Class Weights for Balanced Training}
\begin{tabular}{lrr}
\toprule
\textbf{Risk Bin} & \textbf{Quote Share} & \textbf{Class Weight} \\
\midrule
Low Risk (bin\_0) & 14.9\% & 1.012 \\
Medium Risk (bin\_1) & 16.5\% & 1.012 \\
High Risk (bin\_2) & 68.6\% & 0.977 \\
\bottomrule
\end{tabular}
\end{table}

\textbf{Interpretation}: Medium risk quotes receive the highest weight (1.01) due to their sparsity, while high risk quotes receive lower weights (0.98) to counteract their dominance in the training data.

\subsection{Fold Balance Analysis}

\subsubsection{Case Distribution Across Folds}

\begin{figure}[H]
\centering
\includegraphics[width=0.9\textwidth]{figures/kfold_case_counts.pdf}
\caption{Case count distribution across 3-fold rolling-origin cross-validation. Shows increasing training set size over time with consistent validation/test set sizes.}
\end{figure}

\subsubsection{Quote Distribution by Risk Bin}

\begin{figure}[H]
\centering
\includegraphics[width=0.9\textwidth]{figures/kfold_quote_distribution.pdf}
\caption{Quote distribution across folds colored by risk bin. Shows preservation of quote-level imbalance within each temporal slice, handled via class weighting and case support weights rather than stratification.}
\end{figure}

\subsection{Temporal Balance Assessment}

\begin{figure}[H]
\centering
\includegraphics[width=0.9\textwidth]{figures/kfold_stratification_quality.pdf}
\caption{Case distribution by risk bin across temporal folds. Shows maintenance of approximately balanced case-level distributions within each training window, with quote-level imbalance handled via weighting.}
\end{figure}

\subsection{Case Size Variation Analysis}

\begin{figure}[H]
\centering
\includegraphics[width=0.8\textwidth]{figures/kfold_quotes_per_case.pdf}
\caption{Distribution of average quotes per case across different splits and folds. Box plots reveal variation in case sizes but demonstrate no systematic bias toward larger or smaller cases in any particular split.}
\end{figure}

\subsection{Temporal Fold Statistics Summary}

\begin{table}[H]
\centering
\caption{Rolling-Origin Temporal CV Statistics}
\begin{tabular}{lrrrrr}
\toprule
\textbf{Fold} & \textbf{Train Cases} & \textbf{Val Cases} & \textbf{Test Cases} & \textbf{Train Quotes} & \textbf{Test Quotes} \\
\midrule
Fold 0 & 40 & 10 & 20 & 5,000 & 2,000 \\
Fold 1 & 60 & 10 & 20 & 8,000 & 2,000 \\
Fold 2 & 80 & 10 & 20 & 11,000 & 2,000 \\
\midrule
\textbf{Final Training} & \textbf{100} & \textbf{--} & \textbf{--} & \textbf{22,500} & \textbf{--} \\
\textbf{OOF Test} & \textbf{--} & \textbf{--} & \textbf{25} & \textbf{--} & \textbf{1,716} \\

\bottomrule
\end{tabular}
\end{table}

\textbf{Note}: Final training fold combines all CV data for production model training. OOF test provides final evaluation on temporally held-out data.

\subsection{Cross-Validation Validation}

The rolling-origin temporal approach successfully addresses several key challenges:

\begin{itemize}
\item \textbf{Data Leakage Prevention}: No case appears in multiple folds, ensuring true out-of-sample evaluation
\item \textbf{Temporal Purity}: Training data temporally precedes evaluation data, preventing future information leakage
\item \textbf{Per-Fold Tertile Consistency}: Train-only boundaries ensure realistic evaluation on each temporal slice
\item \textbf{Statistical Power}: Each fold contains sufficient examples for robust evaluation within temporal constraints
\item \textbf{Distribution Shift Detection}: OOF test shows expected prior shift (Medium: 722→978, High: 692→432) indicating realistic temporal evolution
\end{itemize}

This rigorous cross-validation framework ensures that model performance estimates are both unbiased and generalizable to future legal cases, while acknowledging natural distribution drift over time.

\subsection{Distribution Shift and Temporal Evolution}

The OOF test set demonstrates expected temporal distribution shift compared to the metadata baseline expectations:
\begin{itemize}
\item \textbf{Medium class}: Expected 722 quotes → Observed 978 quotes (+256, +35\%)
\item \textbf{High class}: Expected 692 quotes → Observed 432 quotes (-260, -38\%)
\item \textbf{Low class}: Expected 302 quotes → Observed 306 quotes (+4, +1\%)
\end{itemize}

This shift reflects realistic evolution in litigation patterns over time and validates our temporal methodology's ability to detect and measure distribution drift in the legal domain.

\subsection{Model Monitoring and Limitations}

\textbf{Distribution Monitoring}: We monitor Population Stability Index (PSI) for all features quarterly. Features exceeding PSI > 0.25 trigger model recalibration or feature demotion protocols.

\textbf{Temporal Associations vs. Causality}: Our methodology establishes temporal associations between linguistic features and litigation outcomes. These are not causal relationships, and the model should be interpreted within the broader context of legal and economic factors affecting litigation severity.

\begin{figure}[H]
\centering
\includegraphics[width=1.0\textwidth]{figures/temporal_holdouts_across_folds.pdf}
\caption{Temporal holdouts across folds showing rolling origin cross-validation design. Each fold's training data temporally precedes its evaluation data, ensuring no temporal leakage. The final training fold combines all CV data for the final model training.}
\end{figure}


\end{document}
