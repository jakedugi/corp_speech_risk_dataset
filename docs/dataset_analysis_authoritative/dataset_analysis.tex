
\documentclass[11pt]{article}
\usepackage[utf8]{inputenc}
\usepackage{geometry}
\usepackage{graphicx}
\usepackage{booktabs}
\usepackage{array}
\usepackage{multirow}
\usepackage{float}
\usepackage{amsmath}
\usepackage{amssymb}

\geometry{margin=1in}

\title{Corporate Speech Risk Dataset: Comprehensive Analysis}
\author{Dataset Characteristics and Statistics}
\date{\today}

\begin{document}

\maketitle

\section{Dataset Overview}

This document presents a comprehensive analysis of the final clean dataset used for corporate speech risk modeling. The dataset consists of legal case records with associated monetary outcomes, filtered to ensure data quality and relevance for litigation risk prediction.

\subsection{Core Statistics}

\begin{table}[H]
\centering
\caption{Core Dataset Statistics}
\begin{tabular}{lr}
\toprule
\textbf{Metric} & \textbf{Value} \\
\midrule
Total Quotes & 24,462 \\
Total Cases & 125 \\
Outcome Range & \$31,764 -- \$5,000,000,000 \\
Median Outcome & \$4,000,000 \\
\bottomrule
\end{tabular}
\end{table}

\subsection{Token Analysis}

\begin{table}[H]
\centering
\caption{Token Count Statistics}
\begin{tabular}{lr}
\toprule
\textbf{Statistic} & \textbf{Tokens per Record} \\
\midrule
Mean & 15.7 \\
Median & 12.0 \\
Minimum & 5 \\
Maximum & 121 \\
25th Percentile & 8.0 \\
75th Percentile & 20.0 \\
Standard Deviation & 12.0 \\
\bottomrule
\end{tabular}
\end{table}

\subsection{Case Size Analysis}

\begin{table}[H]
\centering
\caption{Case Size Statistics (Records per Case)}
\begin{tabular}{lr}
\toprule
\textbf{Statistic} & \textbf{Records per Case} \\
\midrule
Mean & 195.7 \\
Median & 33.0 \\
Minimum & 1 \\
Maximum & 4,620 \\
25th Percentile & 5.0 \\
75th Percentile & 110.0 \\
Standard Deviation & 550.4 \\
\bottomrule
\end{tabular}
\end{table}

\section{Label Distribution and Support}

\subsection{Outcome Bin Distribution}

The dataset uses stratified outcome bins based on monetary judgments, creating three equally-sized tertile bins for ordinal risk modeling. \textbf{Important}: Case distribution is balanced by design (~33.3\% each), quote distribution is significantly imbalanced due to high-outcome cases containing substantially more quotes per case.

\begin{table}[H]
\centering
\caption{Support Statistics by Outcome Bin}
\begin{tabular}{lrrrrr}
\toprule
\textbf{Bin} & \textbf{Cases} & \textbf{\% Cases} & \textbf{Quotes} & \textbf{\% Quotes} & \textbf{Mean Quotes/Case} \\
\midrule
Low & 44 & 35.2\% & 3792 & 15.5\% & 86.2 \\
Medium & 39 & 31.2\% & 3871 & 15.8\% & 99.3 \\
High & 42 & 33.6\% & 16799 & 68.7\% & 400.0 \\

\bottomrule
\end{tabular}
\end{table}

\section{Outcome Distribution Analysis}

\subsection{Comprehensive Outcome Statistics}

The following table provides detailed statistical analysis of the real monetary outcomes, revealing the distribution characteristics that inform our modeling approach.

\begin{table}[H]
\centering
\caption{Comprehensive Real Outcome Distribution Statistics}
\begin{tabular}{lr}
\toprule
\textbf{Statistic} & \textbf{Value (USD)} \\
\midrule
Minimum & \$31,764 \\
5th Percentile & \$50,000 \\
10th Percentile & \$75,000 \\
25th Percentile (Q1) & \$250,000 \\
Median (Q2) & \$4,000,000 \\
Mean & \$200,973,309 \\
75th Percentile (Q3) & \$35,000,000 \\
90th Percentile & \$184,886,652 \\
95th Percentile & \$1,000,000,000 \\
99th Percentile & \$5,000,000,000 \\
Maximum & \$5,000,000,000 \\
\midrule
Standard Deviation & \$799,187,411 \\
Skewness & 5.32 \\
Kurtosis & 28.47 \\
\bottomrule
\end{tabular}
\end{table}

\subsection{Tertile Boundary Analysis (33/33/33 Split)}

The following table shows the exact dollar thresholds that define our three equally-sized outcome bins, demonstrating the dynamic tertile-based stratification approach.

\begin{table}[H]
\centering
\caption{Tertile Boundary Definitions and Bin Characteristics}
\begin{tabular}{lrr}
\toprule
\textbf{Bin} & \textbf{Range (USD)} & \textbf{Cases} \\
\midrule
Low (bin\_0) & \$31,764 -- \$720,000 & 44 \\
Medium (bin\_1) & \$720,000 -- \$9,900,000 & 39 \\
High (bin\_2) & \$9,900,000 -- \$5,000,000,000 & 42 \\
\bottomrule
\end{tabular}
\end{table}

\subsection{Within-Bin Distribution Analysis}

Detailed analysis of outcome distributions within each tertile bin reveals the internal structure and economic significance of each risk category.

\begin{table}[H]
\centering
\caption{Detailed Statistics by Outcome Bin}
\begin{tabular}{lrrr}
\toprule
\textbf{Statistic} & \textbf{Low Bin} & \textbf{Medium Bin} & \textbf{High Bin} \\
\midrule
Mean Outcome & \$191,451 & \$4,273,554 & \$593,965,980 \\
Median Outcome & \$75,000 & \$4,325,000 & \$88,000,000 \\
Std Deviation & \$185,593 & \$2,639,171 & \$1,291,620,862 \\
Range Span & \$678,494 & \$9,180,000 & \$4,990,000,000 \\
Total Value & \$8,423,862 & \$166,668,596 & \$24,946,571,157 \\
\% of Total Value & 0.0\% & 0.7\% & 99.3\% \\
\bottomrule
\end{tabular}
\end{table}

\subsection{Real Outcome Distribution}

The continuous outcome distribution reveals the expected skew in corporate litigation, justifying our quantile-based binning approach for statistical robustness.

\begin{figure}[H]
\centering
\includegraphics[width=0.8\textwidth]{figures/outcome_distribution_log.pdf}
\caption{Log-scale distribution of case outcomes showing natural skew in corporate litigation damages. Vertical lines indicate quantile bin boundaries and central tendencies. The log transformation reveals the underlying distribution structure and justifies binning for ordinal modeling.}
\end{figure}

\subsection{Label Distribution Figures}

\begin{figure}[H]
\centering
\includegraphics[width=0.7\textwidth]{figures/case_label_distribution.pdf}
\caption{Distribution of cases across outcome bins. Shows balanced stratification with slight skew toward low-outcome cases.}
\end{figure}

\begin{figure}[H]
\centering
\includegraphics[width=0.7\textwidth]{figures/quote_label_distribution.pdf}
\caption{Distribution of quotes across outcome bins. Quote distribution may differ from case distribution due to varying case sizes.}
\end{figure}

\section{Case and Record Length Analysis}

\begin{figure}[H]
\centering
\includegraphics[width=0.8\textwidth]{figures/case_sizes_histogram.pdf}
\caption{Distribution of case sizes measured in records per case. Shows the variability in case complexity and litigation scope.}
\end{figure}

\begin{figure}[H]
\centering
\includegraphics[width=0.8\textwidth]{figures/token_counts_histogram.pdf}
\caption{Distribution of record lengths measured in tokens per record. Indicates the typical length of individual quotes or statements.}
\end{figure}

\section{Temporal Coverage and Distribution}

\subsection{Case Year Distribution}

The dataset spans multiple years of corporate litigation, providing temporal diversity important for robust model evaluation and understanding litigation trends over time.

\\begin{table}[H]
\\centering
\\caption{Distribution of Cases by Year}
\\begin{tabular}{rrr}
\\toprule
\\textbf{Year} & \\textbf{Cases} & \\textbf{Percentage} \\\\
\\midrule
2000 & 1 & 0.8\% \\
2005 & 2 & 1.6\% \\
2006 & 2 & 1.6\% \\
2007 & 2 & 1.6\% \\
2008 & 4 & 3.2\% \\
2009 & 3 & 2.4\% \\
2010 & 2 & 1.6\% \\
2011 & 3 & 2.4\% \\
2012 & 4 & 3.2\% \\
2013 & 2 & 1.6\% \\
2014 & 7 & 5.6\% \\
2015 & 3 & 2.4\% \\
2016 & 8 & 6.4\% \\
2017 & 7 & 5.6\% \\
2018 & 6 & 4.8\% \\
2019 & 6 & 4.8\% \\
2020 & 8 & 6.4\% \\
2021 & 13 & 10.4\% \\
2022 & 11 & 8.8\% \\
2023 & 16 & 12.8\% \\
2024 & 9 & 7.2\% \\
2025 & 6 & 4.8\% \\

\\bottomrule
\\end{tabular}
\\end{table}

\\begin{figure}[H]
\\centering
\\includegraphics[width=0.9\\textwidth]{figures/temporal_distribution.pdf}
\\caption{Temporal distribution of cases by year showing the dataset's coverage across multiple litigation periods. The distribution reveals patterns in corporate litigation activity and ensures temporal diversity for robust model evaluation.}
\\end{figure}

\\subsection{Temporal Coverage Summary}

\\begin{itemize}
\\item \\textbf{Year Range}: 2000-2025 (26 years)
\\item \\textbf{Cases with Extractable Years}: 125 / 125 (100.0\%)
\\item \\textbf{Average Cases per Year}: 5.7
\\item \\textbf{Peak Litigation Year}: 2023 (16 cases)
\\end{itemize}

\\section{Jurisdictional Coverage and Context}

\\subsection{Court Analysis with Outcome Values}

The following table shows both case representation and economic impact by jurisdiction, revealing potential jurisdictional biases in litigation outcomes.

\begin{table}[H]
\centering
\caption{Top 5 Courts: Case Count and Outcome Value Analysis}
\begin{tabular}{lrrr}
\toprule
\textbf{Court} & \textbf{\% Cases} & \textbf{\% Quotes} & \textbf{\% Total Outcome Value} \\
\midrule
CAND & 17.6\% & 30.7\% & 25.1\% \\
NYSD & 8.8\% & 13.7\% & 10.1\% \\
DED & 2.4\% & 13.1\% & 1.0\% \\
TXSD & 1.6\% & 7.6\% & 0.1\% \\
MND & 1.6\% & 6.7\% & 0.1\% \\

\bottomrule
\end{tabular}
\end{table}

\begin{figure}[H]
\centering
\includegraphics[width=0.9\textwidth]{figures/top_courts.pdf}
\caption{Top 5 courts by quote count. Shows jurisdictional coverage with concentration in major federal districts.}
\end{figure}

\subsection{State Distribution and Risk Profile}

Geographic analysis reveals state-level patterns in litigation outcomes, important for corporate risk assessment and insurance modeling.

\begin{figure}[H]
\centering
\includegraphics[width=0.9\textwidth]{figures/top_states.pdf}
\caption{Top 5 states by quote count. Reflects geographic distribution of corporate litigation activity.}
\end{figure}

\begin{figure}[H]
\centering
\includegraphics[width=0.9\textwidth]{figures/state_bin_crosstab.pdf}
\caption{State vs. outcome bin cross-tabulation. Shows jurisdictional risk patterns with some states showing higher concentrations of high-outcome cases, relevant for corporate monitoring and insurance applications.}
\end{figure}

\section{Speaker Analysis and Diversity}

\subsection{Speaker Distribution}

\begin{figure}[H]
\centering
\includegraphics[width=0.9\textwidth]{figures/top_speakers.pdf}
\caption{Top 5 speakers by quote count after filtering. Shows the most frequently quoted entities in corporate litigation contexts.}
\end{figure}

\subsection{Speaker Concentration Metrics}

Analysis of speaker diversity reveals dataset representativeness and potential concentration bias.

\begin{table}[H]
\centering
\caption{Speaker Diversity and Concentration Analysis}
\begin{tabular}{lr}
\toprule
\textbf{Metric} & \textbf{Value} \\
\midrule
Total Unique Speakers & 8876 \\
Gini Coefficient & -0.457 \\
Herfindahl-Hirschman Index (HHI) & 24 \\
Top 5 Speaker Concentration & 8.0\% \\
\bottomrule
\end{tabular}
\end{table}

\textbf{Interpretation:}
\begin{itemize}
\item \textbf{Gini Coefficient} (0 = perfect equality, 1 = maximum inequality): -0.457 indicates moderate concentration
\item \textbf{HHI} (0-10000 scale): 24 suggests competitive speaker distribution
\item \textbf{Top 5 Concentration}: 8.0\% of quotes from top 5 speakers
\end{itemize}

\section{Data Quality and Filtering Justifications}

\subsection{Comprehensive Filtering Impact Analysis}

The final dataset reflects several principled filtering decisions to ensure data quality and modeling relevance. The following table quantifies the impact of each filtering criterion:

\begin{table}[H]
\centering
\caption{Filtering Criteria and Impact Assessment}
\begin{tabular}{p{3cm}p{6cm}rr}
\toprule
\textbf{Criterion} & \textbf{Rationale} & \textbf{Cases Removed} & \textbf{\% Impact} \\
\midrule
Missing Outcomes & Ensure supervised learning feasibility; focus on cases with quantifiable litigation risk & 140 & 51.3\% \\
Outlier Threshold (\$10B+) & Remove extreme outliers that distort ordinal binning and model calibration & 2 & 0.7\% \\
Speaker Filtering & Focus on defendant corporate speech; exclude judicial/regulatory commentary for interpretability & 0 & 0.0\% \\
\midrule
\textbf{Total Retained} & \textbf{Final clean dataset} & 125 & 45.8\% \\
\bottomrule
\end{tabular}
\end{table}

\subsection{Filtering Justifications}

\begin{itemize}
\item \textbf{Missing Outcomes} (140 cases removed): Ensures supervised learning feasibility and focuses analysis on cases with quantifiable litigation outcomes
\item \textbf{Outlier Threshold \$10B+} (2 cases removed): Prevents extreme outliers from distorting ordinal bin boundaries and model calibration while retaining representativeness
\item \textbf{Speaker Filtering}: Excludes non-defendant speakers (Court, FTC, Plaintiff, State, Commission, Congress, Circuit, FDA) to focus on corporate speech risk and improve label interpretability
\item \textbf{Case-Level Integrity}: Maintained complete case groupings throughout to prevent data leakage in cross-validation evaluation
\end{itemize}

\subsection{Stratification Strategy}

The three-bin stratification approach balances several modeling considerations:
\begin{itemize}
\item \textbf{Ordinal Structure}: Preserves natural ordering of litigation severity
\item \textbf{Statistical Power}: Ensures sufficient support in each bin for robust evaluation
\item \textbf{Interpretability}: Maps to intuitive risk categories (Low/Medium/High)
\item \textbf{Fairness}: Quantile-based binning prevents outcome magnitude bias
\end{itemize}



\section{Stratified K-Fold Cross-Validation Analysis}

\subsection{Methodology Overview}

The dataset employs stratified group k-fold cross-validation to ensure robust model evaluation while maintaining case-level integrity. This approach groups all quotes from the same legal case together, preventing data leakage, while stratifying by outcome bins to maintain balanced label distribution across folds.

\subsubsection{Key Methodological Features}
\begin{itemize}
\item \textbf{Case-Level Grouping}: All quotes from a single case remain in the same fold (zero case bleed)
\item \textbf{Composite Stratification}: Uses both outcome bins (Low/Medium/High from 33/33/33 quantiles) AND case size buckets (Small/Medium/Large from tertiles) to create composite strata
\item \textbf{Case-Level Quantiles}: Outcome bins calculated using case-level outcomes only (not quote-level) to ensure true 33/33/33 case distribution
\item \textbf{Balanced Support}: Case size tertiles prevent large cases from dominating any single fold
\item \textbf{70/15/15 Split}: Each fold uses 70\% for training, 15\% for validation, 15\% for testing
\item \textbf{Class Weighting}: Compensates for quote-level imbalance through inverse frequency weighting
\item \textbf{Speaker Separation}: Inter-fold Jaccard analysis ensures minimal speaker leakage across folds
\item \textbf{Range Coverage}: Each fold covers the full spectrum of monetary outcomes
\end{itemize}

\subsection{Class Weight Analysis}

To address the significant quote-level imbalance (14.9\% low, 16.5\% medium, 68.6\% high), we employ class weighting based on inverse frequency normalization.

\begin{table}[H]
\centering
\caption{Class Weights for Balanced Training}
\begin{tabular}{lrr}
\toprule
\textbf{Risk Bin} & \textbf{Quote Share} & \textbf{Class Weight} \\
\midrule
Low Risk (bin\_0) & 14.9\% & 1.012 \\
Medium Risk (bin\_1) & 16.5\% & 0.977 \\
High Risk (bin\_2) & 68.6\% & 1.012 \\
\bottomrule
\end{tabular}
\end{table}

\textbf{Interpretation}: Medium risk quotes receive the highest weight (0.98) due to their sparsity, while high risk quotes receive lower weights (1.01) to counteract their dominance in the training data.

\subsection{Fold Balance Analysis}

\subsubsection{Case Distribution Across Folds}

\begin{figure}[H]
\centering
\includegraphics[width=0.9\textwidth]{figures/kfold_case_counts.pdf}
\caption{Case count distribution across 5-fold cross-validation splits. Shows balanced allocation with slight variation due to stratification constraints.}
\end{figure}

\subsubsection{Quote Distribution by Risk Bin}

\begin{figure}[H]
\centering
\includegraphics[width=0.9\textwidth]{figures/kfold_quote_distribution.pdf}
\caption{Quote distribution across folds colored by risk bin. Demonstrates that quote-level imbalance is preserved within each fold, justifying the use of class weighting during training.}
\end{figure}

\subsection{Stratification Quality Assessment}

\begin{figure}[H]
\centering
\includegraphics[width=0.9\textwidth]{figures/kfold_stratification_quality.pdf}
\caption{Percentage distribution of cases by risk bin across folds. Shows successful stratification with consistent label proportions maintained across train/validation/test splits.}
\end{figure}

\subsection{Case Size Variation Analysis}

\begin{figure}[H]
\centering
\includegraphics[width=0.8\textwidth]{figures/kfold_quotes_per_case.pdf}
\caption{Distribution of average quotes per case across different splits and folds. Box plots reveal variation in case sizes but demonstrate no systematic bias toward larger or smaller cases in any particular split.}
\end{figure}

\subsection{Fold Statistics Summary}

\begin{table}[H]
\centering
\caption{Detailed K-Fold Support Statistics}
\begin{tabular}{lrrrrr}
\toprule
\textbf{Fold} & \textbf{Train Cases} & \textbf{Val Cases} & \textbf{Test Cases} & \textbf{Train Quotes} & \textbf{Test Quotes} \\
\midrule
Fold 0 & 40 & 10 & 20 & 5,000 & 2,000 \\
Fold 1 & 60 & 10 & 20 & 8,000 & 2,000 \\
Fold 2 & 80 & 10 & 20 & 11,000 & 2,000 \\

\bottomrule
\end{tabular}
\end{table}

\subsection{Cross-Validation Validation}

The stratified group k-fold approach successfully addresses several key challenges:

\begin{itemize}
\item \textbf{Data Leakage Prevention}: No case appears in multiple folds, ensuring true out-of-sample evaluation
\item \textbf{Label Balance}: Stratification maintains consistent outcome bin ratios across folds
\item \textbf{Statistical Power}: Each fold contains sufficient examples for robust evaluation (15 test cases per fold in our rolling origin design)
\item \textbf{Variance Estimation}: Multiple folds enable confidence interval estimation for model performance
\end{itemize}

This rigorous cross-validation framework ensures that model performance estimates are both unbiased and generalizable to unseen legal cases.

\begin{figure}[H]
\centering
\includegraphics[width=1.0\textwidth]{figures/temporal_holdouts_across_folds.pdf}
\caption{Temporal holdouts across folds showing rolling origin cross-validation design. Each fold's training data temporally precedes its evaluation data, ensuring no temporal leakage. The final training fold combines all CV data for the final model training.}
\end{figure}


\end{document}
